\documentclass[12pt,a4paper]{report}
\usepackage[hmargin=3cm,vmargin=3cm]{geometry}
\usepackage{graphicx}
\usepackage{caption}
\usepackage{gensymb}
\usepackage{array}
\usepackage{ragged2e}
\usepackage{listings}
\graphicspath{Images}
\usepackage{hyperref}
\usepackage{subcaption}
\usepackage{float}
\usepackage{listings}
\usepackage{xcolor}

\hypersetup{
	colorlinks,
	citecolor=black,
	filecolor=black,
	linkcolor=darkgray,
	urlcolor=blue
}

\definecolor{codegreen}{rgb}{0,0.6,0}
\definecolor{codegray}{rgb}{0.5,0.5,0.5}
\definecolor{codepurple}{rgb}{0.58,0,0.82}
\definecolor{backcolour}{rgb}{0.95,0.95,0.92}

\lstdefinestyle{mystyle}{
	backgroundcolor=\color{backcolour},   
	commentstyle=\color{codegreen},
	keywordstyle=\color{magenta},
	numberstyle=\tiny\color{codegray},
	stringstyle=\color{codepurple},
	basicstyle=\ttfamily\footnotesize,
	breakatwhitespace=false,         
	breaklines=true,                 
	captionpos=b,                    
	keepspaces=true,                 
	numbers=left,                    
	numbersep=5pt,                  
	showspaces=false,                
	showstringspaces=false,
	showtabs=false,                  
	tabsize=2
}

\begin{document}

\begin{figure}
	\centering
	\includegraphics[width = 0.3\textwidth]{iit}
	\hspace{1cm}
	\includegraphics[width = 0.4\textwidth]{fossee-logo}
\end{figure}

\title
	{\textbf
			{\textbf
				{Summer Fellowship Report}}
				\vspace{5mm} 
				\\\small On 
				\\\vspace{5mm} 
				\textbf
				{\large PROJECT RECYCLE : AN ANALYSIS OF SOLID WASTE MANAGEMENT SYSTEM IN THIRUVANANTHAPURAM MUNICIPAL CORPORATION}
				\vspace{5mm} 
				\\\vspace{5mm}
				\small Submitted by
				\\\vspace{5mm}  
				\large 
				\textbf
					{TEAM BITPLEASE :}
					\small AJAY RAGH, AKHIL M, ASHIQ MEHMOOD, ASWIN S, MIDHUN C KACHAPPILLY, RAAMESH BHARDWAJ AND VENUGOPAL PAI (MENTOR)	
					\\ \vspace{5mm}
				\small Under the guidance of \\ \vspace{5mm}
				\large 
				\textbf{Prof. Kannan M. Moudgalya} 
				\vspace{1mm}\\ Chemical Engineering Department  \vspace{1mm} \\IIT Bombay
}
\vspace{1cm}


\maketitle


\newpage
\title{\textbf{\textbf{\LARGE 
\begin{flushleft}
\textbf{Acknowledgment}
\end{flushleft}
}}}

We are fortunate to have an amazing mentor, Mr. Venugopal Pai, who has helped us push our limits and guide us whenever we were stuck. Without his tremendous support, we wouldn’t have gotten this far. Therefore, our first and foremost gratitude goes to him. Next, our sincere gratitude and appreciation go to Dr. Kannan M Moudgalya, Professor at the Department of Chemical Engineering, IIT Bombay and the department of FOSSEE for providing us with this wonderful opportunity. Special mention to Ms. Smita Wangikar of team FOSSEE at IIT-B, who has guided us through the procedures of this fellowship program.

We would have never known about this fellowship program if it were not for Mr. Vishnu Easwaran, Research Associate at IIT-B and a mentor to all of us. Special thanks to Mr. Akhil S G and all the members of Sahridaya for their constant support and encouragement. Ms. Deepthi J Patric of ICFOSS, TVM, who helped us with the software QGIS.

This was a data analysis project, which requires the collection of data, which would not have been possible without the help of people at the Thiruvananthapuram Municipal Corporation (TMC) office. Our sincere gratitude goes to the Mayor, Mr. V K Prasanth and the Collector, Dr. K Vasuki IAS. Mr. Anoop Roy, the Health Inspector at TMC office and Mr. Nikhilesh Paliath of Thanal who had enlightened us with the workings of the waste management system. The HI’s of other circles who have provided us with the required data. The technicians who work at the Aerobic bin unit facilities for their excellent cooperation.


\listoffigures


\tableofcontents

\chapter{\textbf{Introduction}}

Solid waste management and disposal are one of the most emerging challenges of the modern world, especially in economically developing countries due to their growing populations, lifestyle change, rising community living standards and increasing waste generation rates with the consequent increase in land requirements for waste disposal and dumping. We were able to understand that the major initiatives in Thiruvananthapuram city over the last 3 years were regarding solid waste management and to turn it into a green city. The following initiatives were taken under the program to drive the change:-

\begin{itemize}
\item Green protocol implementation
\item Decentralized Waste Management
\item Green Army - a platform where individuals and groups work with school students to educate them about  segregated waste management and other sustainable living practices in an urban environment
\item Waste management awareness festivals
\item Zero waste ward initiative
\end{itemize}

From our discussions, we identified a need for an impact analysis of the initiatives and generate performance models that could be used to implement the above initiatives in other districts in Kerala. Multiple innovative solid waste management technologies were also implemented along with the drive-in different parts of Thiruvananthapuram, we had identified that the impact analysis and optimization of the same were other requirements.
In our project, we wish to include:-

\begin{itemize}
\item Data collection from various organisations, government bodies, health inspectors and Green Army volunteers across the city.
\item Engaging with recognised organisations in the field of solid waste management to identify Key Performance Indicators.
\item To do an analysis of existing inflow and outflow, and assist government agencies in identifying gaps and engage in capacity building activities.
\item Identifying and generating models to optimize the new technologies that were implemented as part of the initiative.
\item Generate a template for easy implementation of the above initiatives in other districts.
\end{itemize}

\chapter{\textbf{Scope of Study}}

\section{Thiruvananthapuram City - An Overview}

Thiruvananthapuram is a coastal city located at 8$^\circ$ 30' 30.77" N latitude and 76$^\circ$ 57' 9.37" E longitude along the south-western corner of India. The city is spread over an area of 214.86 km sq. The Average population in the city is nearly 9.60 lakh (as per Census 2011). This population is distributed over 100 wards for administrative purposes (Fig. \ref{fig:wardmapoftvm}). The average population density is 4,470 people per Sq. Km. The Literacy rate city is 86.75\%. The language spoken is Malayalam. Other languages include English, Tamil, Hindi. There are 181 Government schools in the city and about 20 Anganwadis.

\begin{figure}[H]
	\centering
	\includegraphics[width=0.85\linewidth]{ward_map_of_tvm}
	\caption{Ward map of Thiruvananthapuram}
	\label{fig:wardmapoftvm}
\end{figure}

Topographically, the city is not uniform, with elevation gradually rising from sea level in the west. The city is sandwiched between the Western Ghats and the Arabian Sea. The city is built on hills by the seashore. The region can be divided into three geographical regions, the lowlands, the midlands and the highlands. The city has a 28 km long coastline. Ponmudi and Mukkunnimala are hill resorts near the city.
The city has a tropical climate and therefore does not experience distinct seasons. In addition, nearness to the Lakshadweep Sea also modifies the climate by providing a moderate influence. The annual mean maximum and mean minimum temperature in the city are 36.2$^\circ$C and 23$^\circ$C respectively. The annual average rainfall as about 1835 mm (Fig. \ref{fig:kcr}). The rainfall is mainly received during South West monsoon season followed by North-East monsoon and pre-monsoon. The humidity is high and rises to about 90\% during the monsoon season.

\begin{figure}[H]
	\centering
	\includegraphics[width=0.75\linewidth]{kcr}
	\caption{Rainfall pattern in Thiruvananthapuram}
	\label{fig:kcr}
\end{figure}

	
\section{Solid Waste}
As per the Solid Waste Management Rules 2016, “Solid Waste” means and includes solid or semi-solid domestic waste, sanitary waste, commercial waste, institutional waste, catering and market waste and other non-residential waste, street sweepings, silt removed or collected from the surface drains, horticulture waste, agriculture, and dairy waste, treated biomedical waste excluding industrial waste, biomedical waste, and e-waste, battery waste, radioactive waste generated in the area under the local authorities and other entities. Any material which is not needed by the owner, producer or processor. Solid waste comprises all the wastes from human and animal activities that are normally solid and are discarded as useless or unwanted.

\subsection{Types of Solid Wastes}

\begin{itemize}
	\item Municipal Waste - Municipal waste includes wastes resulting from municipal activities and services such as street waste, dead animals, market waste and abandoned vehicles. However, the term is commonly applied in a wider sense to incorporate domestic wastes, institutional wastes and commercial wastes.
	
	\item Industrial Waste -  Industrial solid waste means solid waste generated by manufacturing or industrial processes.
	
	\item E-waste - Electronic waste, or e-waste, is a term for electronic products that have become unwanted, non-working or obsolete, and have essentially reached the end of their useful life.
	
	\item Biomedical waste - Any waste which is generated during the diagnosis, treatment or immunization of human beings or animals or in research activities pertaining thereto or in the production or testing of biologicals is termed biomedical waste.
	
	\item Agricultural waste - Agricultural waste is waste produced as a result of various agricultural operations. It includes manure and other wastes from farms, poultry houses and slaughterhouses; harvest waste; fertilizer run- off from fields; pesticides that enter into water, air or soils; and salt and silt drained from fields.
	
	\item Construction and Demolition waste - Construction and demolition waste" means the waste comprising of building materials, debris and rubble resulting from construction, re-modeling, repair and demolition of any civil structure.
	
\end{itemize}

In our study we focused mainly on Bio-degradable Municipal waste. The solid waste generated in trivandrum comprises of about 60-65\% biodegradable waste. These are suitable for converting into compost which can then be supplied for farming and gardening purposes. After the implementation of decentralized management method, about 60\% of households were mandated to dispose their waste at source level itself. 

\chapter{\textbf{Waste Management Systems}}

The waste management system is broadly classified as centralized and decentralized. Industrialist countries such as the USA and Europe faced the problem of increasing waste production first and hence innovations and technological development of waste management strategies came from Europe and the United States. These innovations were focused mainly on the centralized waste management system.

\section{Centralized}
\label{cent}
In the older central waste management system, Thiruvananthapuram Municipal Corporation (TMC) established a Municipal Solid Waste (MSW) processing plant at the nearby Vilappil Grama Panchayat (VGP), through a private agency. The estimated MSW generation in the Corporation was approximately 250 metric tonnes per day. The operator was required to establish an MSW processing plant of 300 metric tonne capacity. But the plant established by the operator was 156 metric tonnes only. Waste was to be processed through aerobic composting, i.e., conversion of biodegradable waste to soil enricher aerobically in windrows. However, the plant did not adhere to the specifications. Because of this, the conversion efficiency of the plant was reduced from 50 percent to 12 percent. This prolonged the processing period of the waste leading to ineffective utilization of the installed capacity of the plant. Due to improper and inefficient operation of the plant, the number of rejects deposited in the plant premises was about 80 percent of the MSW supplied to the plant. The anaerobic conditions caused bad odor and environmental problems and thus adverse public opinion about the workings of the plant leading to widespread conflict and protests. VGP forcibly closed the plant on 21 December 2011.

The problems of centralized waste disposal systems are many, the waste dumping sites in and around the cities emit methane due to anaerobic digestion of organic discards. And they emit a large volume of carbon-di-oxide when these waste piles are burnt. Burning waste piles and stinking dumpsites are a common scene all over India. Centralized solid waste management systems require transportation of waste which indirectly adds to global warming through the consumption of fossil fuels.

\subsection{Waste Management in India}

SWM is one of the most challenging issues for developing countries that face serious pollution problems caused by the generation of large quantities of solid waste. Waste characterization information helps in planning how to reduce waste, set up recycling programs, and conserve money and resources. Suitability of various treatment methods depends on the physical and chemical characteristics of waste. Estimation of resource requirements for collection, transportation, processing, and disposal requires a correct assessment of the quantity of waste generated per day from different sources and their characteristics.
According to the Central Pollution Control Board (CPCB), the average solid waste generated in India ranges from 0.21 to 0.5 kg per capita per day.
The studies carried out by the National Environmental Engineering Research Institute (NEERI) in Indian cities have revealed that quantum of municipal solid waste(MSW) generation varies between 0.21-0.35 kg/capita/day in the urban centers and it goes up to 0.5 kg/capita/day in large cities (NEERI, 1996).

\begin{figure}[H]
	\centering
	\includegraphics[width=0.65\linewidth]{cent_decent}
	\caption{Difference between centralized and decentralized waste management systems}
	\label{fig:centdecent}
\end{figure}

\section{Decentralized}

In the case of the TMC, up until 2013, the method used was a centralized method and collected waste was mostly dumped into MCW plant at VGP as mentioned in Sec. \ref{cent}. After its shutdown,  the corporation started adopting various strategies to implement decentralised waste management. In the decentralized waste management system, waste is segregated and processed at the source. Composting is a common method used for the treatment of biodegradable waste. Non-biodegradable wastes are collected and made available for recycling processes. Thus, decentralized waste management sets a resource loop which diverts as much waste as possible from filling up landfills, which cause a nuisance to the environment and human beings.

\section{Overflow System}
\label{overflow}
The decentralized model of waste management is an optimal waste management system in which almost all the waste generated is taken away from landfills and properly segregated and recycled at the household level or source-level itself. Decentralization is possible at households where there is abundant space available for the construction of composts for organic waste and disposal of other wastes. But if it's not the case then waste starts overflowing out of the household area. Thus a mechanism of overflow waste management is set up,  in which all the waste which comes in excess after disposal at households is collected by the authority and is properly taken care of.

The major problem with households in the TMC area is that they are densely packed. Building a compost requires space and nowadays with increasing population density in the urban areas it's hard for the waste processing at the source level. Thus, in Thiruvananthapuram, an overflow method of waste management is followed, in which after segregation and management at the source level, the excess is collected by the TMC and processed.

Today solid waste management is taken from a different perspective of resource use and management in the background of climate change challenge. Zero Waste is the new approach adopted by many governments across the world and a campaign is on “Zero Waste for Zero Warming” which essentially focusing on eliminating the sources of GHGs and sinking the GHGs.

The Municipal Corporation of Thiruvananthapuram is trying to implement principles of a carbon-neutral city. When the city stopped its centralized solid waste management facility at Vilappilsala and launched source-level management of organic waste, it could save about Rs. one crore per year on diesel consumption for transportation of waste. This is supported further by incorporating principles of urban agriculture with decentralized waste management principles.

\subsection{Aerobic Bin}
Aerobic bin (Fig. \ref{fig:aerobins}) is a 4ft x 4ft x 4ft ferro-cement structure used for aerobic composting of biodegradable waste on community level with minimum of 15 household. With the help of aerobic bins, biodegradable waste can be converted into compost, which can then be used for soil conditioning for farming and gardening. In addition to converting the biodegradable waste to non toxic nutrients, aerobic bins don’t pose any associated issues to local residents. Effective maintenance of Aerobic bins would not cause the emission of foul odor and moreover, the heat produced during the process kills pathogens. Aerobic bins are constructed in such a way that it enables proper air circulation, proper draining of leachate and an easy way of collecting biodegradable waste in layers. Furthermore, no turning or mixing of the equipment is required.

\begin{figure}[H]
	\centering
	\includegraphics[width=1\linewidth]{aerobins}
	\caption{An aerobic bin site}
	\label{fig:aerobins}
\end{figure}

\chapter{\textbf{Data Collection Process}}
The existing method of waste management is entrusted with the health department of Thiruvananthapuram Municipal Corporation(TMC). Initially, our team met with the Health Inspectors (H.I) at TMC office. They gave us a clear idea of the current objectives of the method followed and existing technologies. Also we learned that the TMC was divided into 25 circles and each circle was under the supervision of an assigned H.I.

\begin{figure}[H]
	\centering
	\includegraphics[width=0.7\linewidth]{data_collect}
	\caption{Data collection in process}
	\label{fig:datacollect}
\end{figure}

Our team went to each H.I in-charge of the aerobic bin unit and Material Recovery Facility(MRF) of a particular ward to ask permission for accessing the data available. Further we went to each facility which was maintained by technicians working in shifts, to acquire the information regarding the waste inflow, waste processing rate and outflow (Fig. \ref{fig:datacollect}). All the inflow and outflow of waste of that particular day was recorded in a ledger book. Along with this process, we used a tool called Open Data Kit (ODK) (Fig. \ref{fig:odk} and \ref{fig:odkbuild}) to record the information such as ward name,ward number, latitude and longitude of that facility.

\begin{figure}[H]
	\centering
	\includegraphics[width=0.5\linewidth]{odk}
	\caption{ODK collect interface}
	\label{fig:odk}
\end{figure}	

\begin{figure}[H]
	\centering
	\includegraphics[width=1\linewidth]{ODKbuild}
	\caption{Web interface for ODK Build}
	\label{fig:odkbuild}
\end{figure}	

\chapter{\textbf{Data Analysis}}

After completing the data collection process, it was sorted for analysis purposes. Data from the ledger book was manually entered in LibreOffice Calc and then imported to Rstudio as Comma Separated Values(.CSV) files. The recorded information from ODK was then exported as CSV files and then imported into QGIS. QGIS is a free and open-source cross-platform desktop geographic information system application that supports viewing, editing, and analysis of geospatial data. It helps us to compose a map according to the information recorded from the ODK tool. It allows the person using this map to find out the nearest location of the aerobic bins.

Rstudio allowed us to analyze and create visuals of the CSV files with ease. Different libraries that improved the functionality for plotting interactive graphs include:

\begin{itemize}
	\item ggplot2 - Enables creative plots. The functions such as qplot, ggplot, geomline etc.
	\item plotrix - Enables creating 3D pie chart models
	\item graphics - Enables creating visual graphs and analyse models with built in functions such as barplot, abline, coplot, predict etc
\end{itemize}

Using logical programming methods in R, we analysed the collected data and derived mathematical models so as to study the existing system in Thiruvananthapuram city. Also it was then proposed for other cities to study the feasibility of the model.
	
\section{Waste Inflow and Outflow over Time}

For every ward, the capacity of aerobic is chosen according to the waste generated per day. In an average aerobic bin unit, there are about 10 bins each having a capacity of 1.5 tonnes. Also, the outflow from a particular facility starts only after 30-35 days from its installation period. This is the duration taken for the wet waste stored in the bin to be properly converted to compost which has demand in farming, gardening, etc.

\begin{figure}[H]
	\centering
	\includegraphics[width=0.6\linewidth]{waste_inflow_outflow}
	\begin{subfigure}{.15\textwidth}
		\centering
		\includegraphics[width=1\linewidth]{inflow_outflow_label}
	\end{subfigure}
	\caption{Graph showing waste inflow and outflow}
	\label{fig:wasteinflowoutflow}
\end{figure}

In an average aerobic bin unit, the net inflow of biowaste is about 50kg. After a month and a week, an outflow of a 70 percent rate is observed. This rate sets a balance to prevent the overflow of bin capacity and also maintain a specific amount of compost as inoculum for further stages of decomposition. 

\begin{figure}[H]
	\centering
	\includegraphics[width=0.7\linewidth]{abc_ward}
	\caption{Sample representation of collected data}
	\label{fig:abcward}
\end{figure}

In Fig. \ref{fig:wasteinflowoutflow}), we can see a steep rise in graphical line till the week at which outflow starts, and this is represented using a yellow horizontal line. Following the 6th week, the inclination reduces and continues in a positive direction since there is a remaining portion of 30 percent after outflow which gets added to the subsequent inflow. The green vertical line represents the week till which existing data is available. By extrapolating the graphical line, we could predict the overflow period of an average aerobic bin unit. The red vertical line shows the predicted week at which an average capacity of aerobic bin facility is less than the net waste processing rate and thus waste overflows.



\section{Waste Processing Capability}

The waste processing capability of each of the wards was estimated during the project (Fig. \ref{fig:abcward}). The metric used for estimating the same was a ratio of the amount of waste processed per day in the facilities in these wards to the total amount of waste produced per day by the total population of the wards. The amount of waste generated per capita per day was assumed to be around 0.4 kg (NEERI, 1996).
The total amount of waste processed per day by the different facilities was found from the data in “Solid Waste Management-DPR” which was consolidated by the corporation of Thiruvananthapuram.

\begin{figure}[H]
	\centering
	\includegraphics[width=1\linewidth]{waste_processing}
	\caption{Pie chart showing waste processing capability of wards}
	\label{fig:wasteprocessing}
\end{figure}

\section{Waste Inflow from Collected Data}

From the data collected from different facilities, the average daily inflow of waste to these facilities were calculated by summing up the total waste collected to the total number of days took to collect this amount of waste. This data was then divided into three groups of facilities with a low, medium and high inflow of waste per day. The facilities having less than 50 Kg of waste inflow per day are categorized as low inflow facilities, The facilities having waste inflow between 50-150 Kg waste inflow per day are categorized as medium inflow facilities and those having more than 150 kg waste inflow per day are taken as high inflow facilities. This data is then represented in  the form of a pie chart as shown in Fig \ref{fig:wasteinflow}. From this pie chart, we can see that the vast majority of the facilities are having only a low inflow of waste per day. Only a total of 4 facilities comes under medium and high inflow facilities. This shows that for a large number of facilities are not having their capacities utilized effectively. 

\begin{figure}[H]
	\centering
	\includegraphics[width=1\linewidth]{waste_inflow}
	\caption{Pie chart showing waste inflow}
	\label{fig:wasteinflow}
\end{figure}

\section{Waste Generation Density}

Here, we have made a simple Repräsentativa of the data regarding the waste generation and population of wards. In the table \ref{fig:tablewastedens}, the details of the wards are given. A graphical representation (Fig. \ref{fig:wastegendens}) is made using this data. On the abscissa, we have the population and on the ordinate, we have the waste generated per day. Each of the dots in the graph represents a ward and the size of the dot represents the waste generated per capita per day, or as we termed it, waste generation density. The color represents the range of the waste generation density; blue being closest to the average waste generated per capita of the city, red being more and green being less than that.

\begin{figure}[H]
	\centering
	\includegraphics[width=1\linewidth]{table_waste_dens}
	\caption{Table showing ward details}
	\label{fig:tablewastedens}
\end{figure}

\begin{figure}[H]
	\centering
	\includegraphics[width=1\linewidth]{waste_gen_dens}
	\caption{Graph showing waste generation density of the wards}
	\label{fig:wastegendens}
\end{figure}

We can see by tracing the blue dots that the waste generated is almost proportional to the population. But as shown by the red and green dots, there are a few exceptions to this case. This is mainly due to the presence of the main market places of the city in the wards denoted by red, and predominantly rural areas with source-level elimination of waste in the wards denoted by green dots.

\section{Working with QGIS}
QGIS (previously known as Quantum GIS) is a free and open-source cross-platform desktop geographic information system (GIS) application that supports viewing, editing, and analysis of geospatial data (Fig. \ref{fig:qgisui}). The mapping of all the collected data, creation of heatmaps and plotting out the required sites of MRFs using the mathematical model all have been carried out in the open-source Geographic Information System (GIS)  software QGIS. The QGIS versions 3.4(Madeira) and 3.8 (Zanzibar) for all the analysis purposes. Open Street Maps was used as the basic map layout for plotting the data. 

\begin{figure}[H]
	\centering
	\includegraphics[width=1\linewidth]{qgis_ui}
	\caption{QGIS interface}
	\label{fig:qgisui}
\end{figure}

\subsection{Heat Maps}
Using the Open street map as the layout in QGIS, we created heat maps of TMC. Fig. \ref{fig:popdens} shows the population density of TMC. The darker red part represents highly dense wards whereas lighter red and orange show wards with lesser population density. The high population regions are mainly the urban city areas as analyzed from the map. 

\begin{figure}[H]
	\centering
	\includegraphics[width=0.6\linewidth]{popdens}
	\caption{Heat map of population density}
	\label{fig:popdens}
\end{figure}

Fig \ref{fig:wastedens} shows the density of waste generation per ward, which is a function of total waste generated to the area of the ward. These were mapped from the collected data. On comparing both maps, we can see that most of the waste generated or the higher density of waste production corresponds to the area of the map with a higher density of population. Thus, we can infer that the waste generated is directly related to the population in an area.

\begin{figure}[H]
	\centering
	\includegraphics[width=0.6\linewidth]{wastedens}
	\caption{Heat map of waste generation density}
	\label{fig:wastedens}
\end{figure}

\chapter{\textbf{Mathematical Model}}
One of our main objectives was to generate a mathematical model for an optimum waste management system. As our scope of the study is on biodegradable solid waste, our model will be optimized only for the management of those wastes. In the current waste management system of TMC, the main way to deal with biodegradable solid waste is by using aerobic bins, the details of which have been mentioned in the earlier sections. The other ways in which the waste is discarded include source-level eliminations like bio compost, biogas plants, etc. which are mentioned in the section \ref{overflow}.

\begin{figure}[H]
	\centering
	\includegraphics[width=0.8\linewidth]{mathmod}
	\caption{Alternating flow chart showing the process of generating the mathematical model}
	\label{fig:mathmod}
\end{figure}

Under this context, our mathematical model would be calculating the number of aerobic bin units required and their respective locations, in each ward. For this calculation, the basic parameters used are; the population of the corporation, the total waste generated by the corporation, the population of each ward coming under the corporation, the capacity of one aerobic bin unit in a day and the effect of source-level elimination of waste within that ward (Fig. \ref{fig:mathmod}). This method was applied in Rstudio to obtain the required mathematical model (subsection : \ref{Rcode}). 

%\begin{figure}[H]
%	\centering
%	\includegraphics[width=1\linewidth]{Rcode}
%	\caption{R code of the mathematical model}
%	\label{fig:rcode}
%\end{figure}

\newpage
\subsection{R code}
\label{Rcode}
\lstset{style=mystyle}
\lstinputlisting[language=R]{barchart.R}

\newpage
\section{Optimal Number of Bins in each Ward}
For the first part, to calculate the required number of aerobic bin units in each ward, we take the total waste generated by the corporation in a day and divide it by its total population, we get the waste generated per capita; this value multiplied by a ward’s population would give us the waste generated by that ward in a day. 

The next step is to reduce the effect of the source-level elimination of waste. As mentioned in the previous sections, a portion of waste is treated at the source level itself. After reducing the effect of this factor in the waste generated by the ward, we need to divide it by the capacity of one aerobic bin unit in one day. This would give us the optimum number of aerobic bin units in that particular ward(Fig. \ref{fig:optimal}). 

If we subtract the number of existing aerobic bin units (Fig. \ref{fig:exis}) from the optimal number, we get the number of units to be constructed in each ward (Fig. \ref{fig:construc}). There are a few exceptions to this model which are the main market places of the city where there is abnormal rise in waste generation. Thus, the Corporation has already built many units in those places.

\begin{figure}[H]
	\centering
	\includegraphics[width=0.7\linewidth]{optimal}
	\caption{Bar graph showing the optimal number of aerobic bin units in each ward. The total number of aerobic bins required for TMC is 236.}
	\label{fig:optimal}
\end{figure}
	
\begin{figure}[H]
	\centering
	\includegraphics[width=0.7\linewidth]{exis}
	\caption{Bar graph showing the existing number of aerobic bin units in each ward. The total number of aerobic bins existing in TMC is 47.}
	\label{fig:exis}
\end{figure}

Another important point to be mentioned here is that we are calculating the optimum number of bin units. An Aerobic bin unit may contain from 4 to 10 Aerobic bins, or even more if required. What we are proposing is the desired number of locations were bin units are to be placed. The number of bins in a unit may be varied according to the waste generation of that particular ward. This would help in ease of access to waste disposal facilities for everyone in the ward.

\begin{figure}[H]
	\centering
	\includegraphics[width=0.7\linewidth]{construc}
	\caption{Bar graph showing the number of aerobic bin units to be constructed in each ward.}
	\label{fig:construc}
\end{figure}

A common question that arises at this point is why we have used population and not area or population density for the calculation. A particular ward may contain a large area and less number of people so the waste generated may not be proportional to the area. Also, a ward may have population density higher than the average population density of the corporation, which would result in an error while calculating the waste generated by the ward. These are the reasons why we have avoided the area and population density and instead used the population.

\section{Suggested Locations of Bins in each Ward}
A plot for the proposed location of aerobic bin units in each ward was made using QGIS. As per the obtained output for the collected data we were able to calculate the number of aerobic bin sites required in each ward area so that it is easily accessible to all the population in the ward and also the waste distribution to be even in each site. Thus mapping out the result in Map for the whole TMC area is shown on the left which provides general insight on the number of bins to be constructed and on the right, it shows a zoomed view of a single ward which is marked in red. The division of the ward and placement of aerobic bin is in such a way that the aerobic bin sites are accessible to the population in an easy way. Area divided is based on the least distance to the Bin. The green area shows a possible region of placement of an Aerobic bin site (Fig. \ref{fig:binloc}).

\begin{figure}[H]
	\centering
	\includegraphics[width=1\linewidth]{bin_loc}
	\caption{Map showing ward division and proposed location of aerobic bin units as per the mathematical model}
	\label{fig:binloc}
\end{figure}

\newpage
\section{\textbf{Application of the Mathematical Model}}
The generated mathematical model was applied to two other Corporations, namely Kochi and Alapuzha. The basic parameters of these Corporations were used to calculate the optimal number of aerobic bin units.

\begin{figure}[H]
	\centering
	\includegraphics[width=0.7\linewidth]{kochi}
	\caption{Bar graph showing the optimal number of aerobic bin units for each ward in Kochi}
	\label{fig:kochi}
\end{figure}

\begin{figure}[H]
	\centering
	\includegraphics[width=0.7\linewidth]{alapuzha}
	\caption{Bar graph showing the optimal number of aerobic bin units for each ward in Alapuzha}
	\label{fig:alapuzha}
\end{figure}


\chapter{\textbf{Inference}}
\begin{itemize}
	\item From the study we conducted on TMC’s waste management system, we were able to analyze the inflow and outflow of waste.
	\item We were able to learn programming in R and usage of Rstudio.
	\item We became familiar with usage of QGIS.
	\item We also generated a mathematical model of TMCs Waste Management Systems and Applied it to two other districts.
	\item Although we wanted to analyze the existing system, by comparing it with the old system of centralized waste management, we couldn’t do so due to authoritative issues in obtaining the old data.
	\item As we did the final report, we were able to learn typesetting in LaTeX.
	\item All our collected data and the analyzed data including the R codes can be found in our Git Repo \href{https://github.com/sahridhaya/Recycle}{Recycle}.
	
\end{itemize}

\newpage


\title{\textbf{\textbf{\LARGE 
\begin{flushleft}
\textbf{Reference}
\end{flushleft}
}}}
\begin{itemize}
\item \href{https://www.researchgate.net/publication/260230170_MUNICIPAL_SOLID_WASTE_MANAGEMENT_IN_INDIA_A_REVIEW_AND_SOME_NEW_RESULTS}{MUNICIPAL SOLID WASTE MANAGEMENT IN INDIA : A REVIEW AND SOME NEW RESULTS} 

\item \href{https://www.cag.org.in/blogs/solid-waste-management-dummies}{Solid waste management}
\item \href{https://www.downtoearth.org.in/blog/waste/india-s-challenges-in-waste-management-56753}{Report by DownToEarth magazine}
\item \href{https://www.cag.org.in/database/centralised-and-decentralised-waste-management}{Centralized and Decentralized waste Management }
\item \href{http://www.corporationoftrivandrum.in/sites/default/files/Rating\%20report-TVM.pdf}{TMC Report}
\item \href{https://www.agker.cag.gov.in/images/LSG2011/chiiib.pdf}{Audit Report (LSGIs) for the year ended March 2011} \href{http://thanal.co.in/uploads/resource/document/standard-operating-procedure-of-community-aerobic-composting-bins-thumburmoozhi-model-44732514.pdf}{Thanal : Operation of aerobic bins}
\item \href{http://www.kerala.climatemps.com/}{Rainfall pattern in TMC}
\end{itemize}
\end{document}


